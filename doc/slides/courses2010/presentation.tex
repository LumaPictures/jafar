\documentclass[compress]{beamer}

\mode<presentation> {
  \usetheme{Berlin}

  \setbeamercovered{transparent}
}

\usepackage{default}
\usepackage{listings}
\usepackage{color}

\definecolor{ColorCodeBackground}{rgb}{0.7,0.7,0.7} 

\lstset{language=C++,backgroundcolor=\color{ColorCodeBackground},
basicstyle=\small,numbers=left}

\title{Jafar, a C/C++ interactive development environnement}
\author{Cyril Roussillon}

\AtBeginSection[] {
  \begin{frame}<beamer>
    \tableofcontents[currentsection]
  \end{frame}
}

\begin{document}

\begin{frame}
  \titlepage
\end{frame}

\begin{frame}
  \frametitle{Content of the course}
  \tableofcontents
\end{frame}

%%%%%%%%%%%%%%%%%%%%%%%%%%%%%%%%%%%%%%%%%%%%%%%%%%%%%%%%%%%%%%%%%%%%%%%%%%%%%%%%
%%%%%%%%%%%%%%%%%%%%%%%%%%%%%%%%%%%%%%%%%%%%%%%%%%%%%%%%%%%%%%%%%%%%%%%%%%%%%%%%
%%%%%%%%%%%%%%%%%%%%%%%%%%%%%%%%%%%%%%%%%%%%%%%%%%%%%%%%%%%%%%%%%%%%%%%%%%%%%%%%

\section{Introduction}
\subsection*{}


\begin{frame}
  \frametitle{Main Goals}
  \begin{itemize}
   \item<1-> Sharing: don't reinvent the wheel ?
    \begin{itemize}
     \item<2-> Algorithms,
     \item<3-> Visualisation,
     \item<4-> Data access...
    \end{itemize}
   \item<5-> Ease the development of \textbf{your software}.
   \item<6-> Provide visibility to software.
   \item<7-> Modular:
    \begin{itemize}
      \item<8-> ease of maintenance
      \item<9-> increased reuse
      \item<10-> = faster development, less bugs
		\end{itemize}
  \end{itemize}
\end{frame}

\begin{frame}
 \frametitle{Main features}
 \begin{itemize}
  \item<1-> Support for C/C++,
  \item<2-> A build system: \textit{cmake},
  \item<3-> Modular environnement,
  \item<4-> Interactive shell: \textit{TCL or Ruby (bindings with Swig)},
  \item<6-> Errors reporting: \textit{C++ exceptions},
  \item<7-> Unit testing: \textit{boost},
  \item<8-> Documentation: \textit{doxygen}.
 \end{itemize}
\end{frame}

\begin{frame}
 \frametitle{Jafar and Genom}
 \begin{itemize}
  \item<1-> Very similar: modules, interactive shell...
  \item<2-> They share the library part of the module\\ (no file reading, no display, ...)
  \item<3-> $\Rightarrow$ Integration in Jafar for easy developement (scripts, demos)
  \item<4-> $\Rightarrow$ Integration in Genom for running on robots
 \end{itemize}
\end{frame}


%%%%%%%%%%%%%%%%%%%%%%%%%%%%%%%%%%%%%%%%%%%%%%%%%%%%%%%%%%%%%%%%%%%%%%%%%%%%%%%%
%%%%%%%%%%%%%%%%%%%%%%%%%%%%%%%%%%%%%%%%%%%%%%%%%%%%%%%%%%%%%%%%%%%%%%%%%%%%%%%%
%%%%%%%%%%%%%%%%%%%%%%%%%%%%%%%%%%%%%%%%%%%%%%%%%%%%%%%%%%%%%%%%%%%%%%%%%%%%%%%%

\section{The core of Jafar}
\subsection{Presentation}


\begin{frame}
 \frametitle{Jafar Structure}
 \begin{itemize}
  \item<1-> Directories
    \begin{itemize}
     \item<2-> bin : various tools
     \item<3-> modules : all the installed modules
     \item<4-> doc : the documentation
     \item<5-> build (or build\_debug and build\_release)
    \end{itemize}
  \item<5-> Git repository
    \begin{itemize}
     \item<6-> ssh://trac.laas.fr/git/robots/jafar/jafar : the core of jafar
     \item<7-> ssh://trac.laas.fr/git/robots/jafar/modules : all the modules
    \end{itemize}
  \item<8-> Documentation
    \begin{itemize}
     \item<9-> \textbf{Wiki}: https://intranet.laas.fr/intranet/robots/wiki/Jafar
     \item<10-> \textbf{Doxygen}: http://homepages.laas.fr/croussil/doc/jafar
     \item<11-> \textbf{Doxygen}: \$JAFAR\_DIR/doc/html/index.html
     \item<12-> \textbf{email}: jafar@laas.fr
    \end{itemize}
 \end{itemize}
\end{frame}

\begin{frame}
 \frametitle{Installation}
 https://intranet.laas.fr/intranet/robots/wiki/Jafar/Installation2
 \begin{itemize}
   \item<1-> Install tools dependencies (doxygen, swig, cmake, ...),
   \item<2-> Install external libraries dependencies (boost, opencv, ...),
   \item<3-> Install LAAS libraries dependencies for some modules (t3d, stereopixel, ...),
   \item<4-> Download Jafar (git clone)
   \item<5-> Configure the environment (\$JAFAR\_DIR, \$LD\_LIBRARY\_PATH, ...)
   \item<6-> Configure the build system (cmake)
 \end{itemize}
\end{frame}

\begin{frame}[fragile]
 \frametitle{The makefiles targets}
  At Jafar root or in the module:
  \begin{itemize}
    \item<1-> \verb|make [all]|
    \item<2-> \verb|make test|
    \item<3-> \verb|make clean|
    \item<4-> \verb|make install|
    \item<5-> \verb|make rebuild_cache| : run cmake again if you added/removed files
  \end{itemize}
  At Jafar root only:
  \begin{itemize}
    \item<6-> \verb|make doc|
  \end{itemize}
\end{frame}

\begin{frame}
 \frametitle{Jafar tools}
 In \$JAFAR\_DIR/bin:
 \begin{itemize}
   \item<1-> \textbf{\texttt{jafar\_checkout}}: download a module
   \item<2-> \textbf{\texttt{jafar\_update}}: update jafar and all modules
   \item<3-> \textbf{\texttt{git pull --rebase}}: update jafar or a module
   \item<4-> \textbf{\texttt{jafar\_status}}: display a summary of pending changes and commits
   \item<5-> \textbf{\texttt{jafar\_create}}: create a new module
   \item<6-> \textbf{\texttt{jafar\_add}}: add a new module to the repository
 \end{itemize}
\end{frame}

%%%%%%%%%%%%%%%%%%%%%%%%%%%%%%%%%%%%%%%%%%%%%%%%%%%%%%%%%%%%%%%%%%%%%%%%%%%%%%%%
%%%%%%%%%%%%%%%%%%%%%%%%%%%%%%%%%%%%%%%%%%%%%%%%%%%%%%%%%%%%%%%%%%%%%%%%%%%%%%%%

\subsection{The kernel module}

\begin{frame}
  \frametitle{Kernel module: Features}
  \begin{itemize}
   \item Debug messages,
   \item Usefull macros,
   \item Configuration file,
   \item Timing tools,
   \item ...
  \end{itemize}
\end{frame}

%%%%%%%%%%%%%%%%%%%%%%%%%%%%%%%%%%%%%%%%%%%%%%%%%%%%%%%%%%%%%%%%%%%%%%%%%%%%%%%%

\begin{frame}[fragile]
  \frametitle{Debug (1/2)}
  \begin{itemize}
    \item<1-> DataLogger : to log data into a file
    \item<2-> Debug macro : JFR\_DEBUG, JFR\_VDEBUG, JFR\_VVDEBUG, JFR\_WARNING, JFR\_ERROR
      \begin{lstlisting}
JFR_DEBUG(u << " + " << v << " = " << (u+v));
D:playmodule/file.cpp:709: Robot state after move [13](0,0,0,1,0,0,0,0,0,0,0,0,0)
      \end{lstlisting}
   \item<3-> JFR\_ASSERT / JFR\_PRED\_ERROR : check that a parameter is correct
      \begin{lstlisting}
int MyFirstClass::div(int u, int v) const
{
  JFR_ASSERT(v != 0, "Can't divide by 0");
  return u / v;
}
      \end{lstlisting}
  \end{itemize}
\end{frame}

\begin{frame}[fragile]
  \frametitle{Debug (2/2)}
      Verbosity level: Off, Trace, Warning,  Debug, VerboseDebug, VeryVerboseDebug
      \begin{lstlisting}
kernel::DebugStream::setDefaultLevel(
  kernel::DebugStream::Debug)
kernel::DebugStream::setLevel("playmodule",
  kernel::DebugStream::VeryVerboseDebug);
kernel::DebugStream::setLevel("playmodule",
  kernel::DebugStream::Off);
      \end{lstlisting}
\end{frame}

%%%%%%%%%%%%%%%%%%%%%%%%%%%%%%%%%%%%%%%%%%%%%%%%%%%%%%%%%%%%%%%%%%%%%%%%%%%%%%%%

\begin{frame}[fragile]
  \frametitle{Usefull macros}
  For instance JFR\_FOREACH:
  \begin{lstlisting}
std::vector< CoolObject > coolObjects;
JFR_FOREACH( CoolObject& coolObject, coolObjects )
{
  coolObject.soSomethingCool();
}
  \end{lstlisting}
  Instead of:
  \begin{lstlisting}
std::vector< CoolObject > coolObjects;
for( std::vector< CoolObject >::iterator it
     = coolObjects.begin();
  it = coolObjects.end(); ++it )
{
  it->soSomethingCool();
}
  \end{lstlisting}
\end{frame}

%%%%%%%%%%%%%%%%%%%%%%%%%%%%%%%%%%%%%%%%%%%%%%%%%%%%%%%%%%%%%%%%%%%%%%%%%%%%%%%%

\begin{frame}[fragile]
  \frametitle{Configuration file (1/3)}
Exemple of configuration file:
\begin{lstlisting}
MyValue: 10
OtherValue: hello
\end{lstlisting}
  
Exemple of code to read file:
\begin{lstlisting}
KeyValueFile configFile;
configFile.readFile( "test.cfg");
int val;
configFile.getItem( "MyValue", val );
std::string val2;
configFile.getItem( "OtherValue", val2 );
\end{lstlisting}


\end{frame}

%%%%%%%%%%%%%%%%%%%%%%%%%%%%%%%%%%%%%%%%%%%%%%%%%%%%%%%%%%%%%%%%%%%%%%%%%%%%%%%%

\begin{frame}[fragile]
  \frametitle{Configuration file (2/3)}
KeyValueFileSave: an object which can save its parameters.
 
\begin{lstlisting}
class CoolAlgorithm : public KeyValueFileSave {
  public:
    virtual void saveKeyValueFile(
    jafar::kernel::KeyValueFile& keyValueFile)
    {
      keyValueFile.setItem("MyParameter", m_parameter );
    }
  public:
    int m_parameter;
};

CoolAlgorithm coolAlgorithm;
coolAlgorithm.save("algo.cfg");

\end{lstlisting}
 
\end{frame}

%%%%%%%%%%%%%%%%%%%%%%%%%%%%%%%%%%%%%%%%%%%%%%%%%%%%%%%%%%%%%%%%%%%%%%%%%%%%%%%%

\begin{frame}[fragile]
  \frametitle{Configuration file (3/3)}
KeyValueFileLoad: an object which can load its parameters.
 
\begin{lstlisting}
class CoolAlgorithm : public KeyValueFileLoad {
  public:
    virtual void loadKeyValueFile(
    jafar::kernel::KeyValueFile const& keyValueFile)
    {
      keyValueFile.getItem("MyParameter", m_parameter );
    }
  public:
    int m_parameter;
};

CoolAlgorithm coolAlgorithm;
coolAlgorithm.load("algo.cfg");
\end{lstlisting}
 
\end{frame}

%%%%%%%%%%%%%%%%%%%%%%%%%%%%%%%%%%%%%%%%%%%%%%%%%%%%%%%%%%%%%%%%%%%%%%%%%%%%%%%%

\begin{frame}[fragile]
  \frametitle{Timing tools}
  \begin{itemize}
    \item<1-> Chrono
\begin{lstlisting}
 Chrono chrono;
 chrono.start();
 // Do some extensive computation
 JFR_DEBUG(chrono.elapsed());
\end{lstlisting}
    \item<2-> Framerate
		\item<3-> ...
  \end{itemize}

\end{frame}


%%%%%%%%%%%%%%%%%%%%%%%%%%%%%%%%%%%%%%%%%%%%%%%%%%%%%%%%%%%%%%%%%%%%%%%%%%%%%%%%
%%%%%%%%%%%%%%%%%%%%%%%%%%%%%%%%%%%%%%%%%%%%%%%%%%%%%%%%%%%%%%%%%%%%%%%%%%%%%%%%

\subsection{Unit tests}

\begin{frame}
  \frametitle{What are unit tests ?}
  From wikipedia: \textit{In computer programming, unit testing is a method of
testing that verifies the individual units of source code are working properly.}

  \begin{itemize}
    \item<1-> Test the behavior of individual functions,
    \item<2-> As much as possible independent tests,
    \item<3-> Automatic.
  \end{itemize}

\end{frame}

\begin{frame}
  \frametitle{Why unit tests are important ?}
  \begin{itemize}
    \item<1-> Make sure your code does what you want it to do,
    \item<2-> Speed up development and optimizations/refactoring,
    \item<3-> Make sure nobody else breaks your feature,
    \item<4-> Tests are documentation.
  \end{itemize}
\end{frame}

\begin{frame}[fragile]
  \frametitle{Write an unit test.}
  Add a file test\_suite/test\_MyFirstClass.cpp :
  \begin{lstlisting}
#include <boost/test/auto_unit_test.hpp>
#include <kernel/jafarTestMacro.hpp>
#include "playmodule/MyFirstClass.hpp"

BOOST_AUTO_TEST_CASE( test_MyFirstClass )
{
  MyFirstClass mfc;
  JFR_CHECK_EQUAL( mfc.add(1, 2), 3 );
}
  \end{lstlisting}
  Then:
  \begin{lstlisting}
make test
  \end{lstlisting}
\end{frame}


%%%%%%%%%%%%%%%%%%%%%%%%%%%%%%%%%%%%%%%%%%%%%%%%%%%%%%%%%%%%%%%%%%%%%%%%%%%%%%%%
%%%%%%%%%%%%%%%%%%%%%%%%%%%%%%%%%%%%%%%%%%%%%%%%%%%%%%%%%%%%%%%%%%%%%%%%%%%%%%%%

\subsection{Documentation}


\begin{frame}[fragile]
  \frametitle{Documentation: a brief introduction to doxygen}
  Comments syntax:
  \begin{lstlisting}
	/** mutliple lines comment */
	/// single line comment
	///< post-code comment
	\end{lstlisting}
	
  General tags:
  \begin{itemize}
    \item \textbf{@ingroup} declare a function to be part of a module
    \item \textbf{@ref} give a reference to an other function/class
  \end{itemize}

  Function tags:
  \begin{itemize}
    \item \textbf{@param} describe a parameter
    \item \textbf{@return} describe the return parameter
  \end{itemize}
\end{frame}

\begin{frame}[fragile]
  \frametitle{Lets document MyFirstClass}
  \begin{lstlisting}
/**
 * This is my first class in Jafar. @ref add is
 * the most important function.
 * @ingroup playmodule
 */
class MyFirstClass {
	int data; ///< this is the data
  public:
    /**
     * This function add two numbers.
     * @param u first number
     * @param v second number
     * @return the addition of u with v
     */
    int add(int u, int v) const;
};
  \end{lstlisting}
\end{frame}


%%%%%%%%%%%%%%%%%%%%%%%%%%%%%%%%%%%%%%%%%%%%%%%%%%%%%%%%%%%%%%%%%%%%%%%%%%%%%%%%
%%%%%%%%%%%%%%%%%%%%%%%%%%%%%%%%%%%%%%%%%%%%%%%%%%%%%%%%%%%%%%%%%%%%%%%%%%%%%%%%
%%%%%%%%%%%%%%%%%%%%%%%%%%%%%%%%%%%%%%%%%%%%%%%%%%%%%%%%%%%%%%%%%%%%%%%%%%%%%%%%

\section{A module in Jafar}
\subsection*{}

\begin{frame}[fragile]
  \frametitle{A Jafar module is...}
  \begin{itemize}
   \item<1-> A C/C++ library
		\begin{itemize}
			\item headers in playmodule/\textbf{include}/playmodule/*.hpp
			\item sources in playmodule/\textbf{src}/*.hpp
		\end{itemize}
   \item<2-> A set of tcl/ruby scripts and/or C/C++ demos\\
		\begin{itemize}
			\item scripts in playmodule/\textbf{macro}/*.rb or *.tcl
			\item demos in playmodule/\textbf{demo\_suite}/demo\_*.cpp
		\end{itemize}
   \item<3-> Documentation
		\begin{itemize}
			\item in playmodule/\textbf{doc}/*.doxy
			\item \textbf{doxygen comments in headers files}
		\end{itemize}
   \item<4-> Unit tests
		\begin{itemize}
			\item in playmodule/\textbf{test\_suite}/test\_*.cpp
		\end{itemize}
  \end{itemize}
\end{frame}

\begin{frame}
  \frametitle{Directory structure}
  \begin{center}
    \includegraphics[width=0.85\textwidth]{../graphics/jafar_module.png}
  \end{center}
\end{frame}

\begin{frame}[fragile]
  \frametitle{How to create a module ?}
  Locally create the module:
  \begin{lstlisting}
cd ${JAFAR_DIR}/modules
../bin/jafar_create playmodule
  \end{lstlisting}
  Commit and push the initial version:
  \begin{lstlisting}
../bin/jafar_add playmodule
  \end{lstlisting}
\end{frame}

\begin{frame}
  \frametitle{A tour inside the new module}
  \begin{itemize}
   \item<1-> \textbf{CMakeLists.txt} (dependencies)
   \item<2-> \textbf{include/playmodule.i} (swig additional wrapping)
   \item<3-> \textbf{include/playmoduleException.hpp} (exceptions)
   \item<4-> \textbf{src/playmoduleException.cpp} (exceptions)
   \item<5-> \textbf{macro/} (macros)
   \item<6-> \textbf{test\_suite/} (unit tests)
   \item<7-> \textbf{demo\_suite/} (demos)
   \item<8-> \textbf{doc/} (documentation)
  \end{itemize}
\end{frame}


\begin{frame}[fragile]
  \frametitle{Compilation}
  \begin{itemize}
    \item<1-> Compile it
      \begin{lstlisting}[language=bash]
make
      \end{lstlisting}
    \item<2-> Load it
      \begin{lstlisting}[language=ruby]
require 'jafar/playmodule'
      \end{lstlisting}
    \item<3-> So what is available...
  \end{itemize}
\end{frame}

\begin{frame}[fragile]
  \frametitle{A first class: header}
  \begin{lstlisting}
#ifndef _MY_FIRST_CLASS_
#define _MY_FIRST_CLASS_
namespace jafar {
  namespace playmodule {
    class MyFirstClass {
      public:
        int add(int u, int v) const;
    };
  }
}
#endif
  \end{lstlisting}
\end{frame}

\begin{frame}[fragile]
  \frametitle{A first class: source}
  \begin{lstlisting}
#include "playmodule/MyFirstClass.hpp"

using namespace jafar::playmodule;

int MyFirstClass::add(int u, int v) const
{
  return u + v;
}
  \end{lstlisting}
\end{frame}

\begin{frame}[fragile]
  \frametitle{Bind it}
  Open file \textit{playmodule.i} and add:
  \begin{lstlisting}
#include "playmodule/MyFirstClass.hpp"
  \end{lstlisting}
  And:
  \begin{lstlisting}
%include "playmodule/MyFirstClass.i"
  \end{lstlisting}
\end{frame}

\begin{frame}[fragile]
  \frametitle{Use it}
  \begin{lstlisting}
require 'jafar/playmodule'
obj = Playmodule::MyFirstClass.new
obj.add( 1, 2)
  \end{lstlisting}
\end{frame}


%%%%%%%%%%%%%%%%%%%%%%%%%%%%%%%%%%%%%%%%%%%%%%%%%%%%%%%%%%%%%%%%%%%%%%%%%%%%%%%%
%%%%%%%%%%%%%%%%%%%%%%%%%%%%%%%%%%%%%%%%%%%%%%%%%%%%%%%%%%%%%%%%%%%%%%%%%%%%%%%%
%%%%%%%%%%%%%%%%%%%%%%%%%%%%%%%%%%%%%%%%%%%%%%%%%%%%%%%%%%%%%%%%%%%%%%%%%%%%%%%%

\section{Available modules}

%%%%%%%%%%%%%%%%%%%%%%%%%%%%%%%%%%%%%%%%%%%%%%%%%%%%%%%%%%%%%%%%%%%%%%%%%%%%%%%%
%%%%%%%%%%%%%%%%%%%%%%%%%%%%%%%%%%%%%%%%%%%%%%%%%%%%%%%%%%%%%%%%%%%%%%%%%%%%%%%%

\subsection{Tools modules: jmath, geom, image, camera, datareader, qdisplay, gdhe}
\begin{frame}
  \frametitle{jmath: Features}
  \begin{itemize}
   \item uses BLAS/LAPACK
   \item matrices and vectors computation
   \item linear solvers
   \item least-square optimization
  \end{itemize}
\end{frame}

%%%%%%%%%%%%%%%%%%%%%%%%%%%%%%%%%%%%%%%%%%%%%%%%%%%%%%%%%%%%%%%%%%%%%%%%%%%%%%%%

\begin{frame}[fragile]
  \frametitle{Create vectors and matrix}
  \begin{itemize}
   \item<1-> Bounded vectors
   \begin{onlyenv}<1-1>
   \begin{lstlisting}
jblas::vec2 vec_2;
jblas::vec3 vec_3;
jblas::vec4 vec_4;
   \end{lstlisting}
   \end{onlyenv}
   \item<2-> Unbounded vectors
    \begin{onlyenv}<2-2>
      \begin{lstlisting}
jblas::vec vec_n( 10 );
      \end{lstlisting}
    \end{onlyenv}
   \item<3-> Bounded matrix
    \begin{onlyenv}<3-3>
      \begin{lstlisting}
jblas::mat22 mat_22;
jblas::mat33 mat_33;
jblas::mat44 mat_44;
      \end{lstlisting}
    \end{onlyenv}
   \item<4-> Unbounded matrix
    \begin{onlyenv}<4-4>
      \begin{lstlisting}
jblas::mat mat_nn(100,400);
      \end{lstlisting}
    \end{onlyenv}
    \item<5-> Zero, Scalar and Identity matrix
    \begin{onlyenv}<5-5>
      \begin{lstlisting}
jblas::mat mat = jblas::zero_mat(5);
jblas::mat mat = jblas::identity_mat(5);
jblas::mat mat = 5.0 * jblas::scalar_mat(5); // Matrix filled with 5.0
      \end{lstlisting}
    \end{onlyenv}
  \end{itemize}
\end{frame}

%%%%%%%%%%%%%%%%%%%%%%%%%%%%%%%%%%%%%%%%%%%%%%%%%%%%%%%%%%%%%%%%%%%%%%%%%%%%%%%%

\begin{frame}[fragile]
  \frametitle{Vectors and matrix operations}
  \begin{itemize}
   \item<1-> Addition, substraction
    \begin{onlyenv}<1-1>
      \begin{lstlisting}
jblas::vec2 v1, v2, v3;
v1 = v1 + v2 - v3;
      \end{lstlisting}
    \end{onlyenv}
   \item<2-> Multiplication
    \begin{onlyenv}<2-2>
      \begin{lstlisting}
jblas::mat m1, m2, m3;
m3 = ublas::prod( m1, m2 );
m3 = ublas::prod( m1,
jblas::mat( ublas::prod( m3, m2 ) );
      \end{lstlisting}
    \end{onlyenv}
    \item<3-> Transposition
    \begin{onlyenv}<3-3>
      \begin{lstlisting}
jblas::mat m4 = ublas::trans(m3);
m4 = ublas::prod( ublas::trans(m2), m1 );
      \end{lstlisting}
    \end{onlyenv}
    \item<4-> Inversion
    \begin{onlyenv}<4-4>
      \begin{lstlisting}
ublasExtra::inv( m4 );
      \end{lstlisting}
    \end{onlyenv}
    \item<5-> Dot and cross product
    \begin{onlyenv}<5-5>
      \begin{lstlisting}
ublas::outer_prod( v1, v2 ); // cross product
ublas::inner_prod( v1, v2 ); // dot product
      \end{lstlisting}
    \end{onlyenv}
    \item<6-> Determinant
    \begin{onlyenv}<6-6>
      \begin{lstlisting}
ublasExtra::det( m4 );
      \end{lstlisting}
    \end{onlyenv}
  \end{itemize}
\end{frame}

%%%%%%%%%%%%%%%%%%%%%%%%%%%%%%%%%%%%%%%%%%%%%%%%%%%%%%%%%%%%%%%%%%%%%%%%%%%%%%%%

\begin{frame}[fragile]
  \frametitle{Symmetric matrix}
  \begin{itemize}
   \item<1-> Bounded symmetric matrix:
    \begin{onlyenv}<1-1>
      \begin{lstlisting}
jblas::sym_mat22 mat_22;
jblas::sym_mat33 mat_33;
jblas::sym_mat44 mat_44;
      \end{lstlisting}
    \end{onlyenv}
   \item<2-> Unbounded symmetric matrix:
    \begin{onlyenv}<2-2>
      \begin{lstlisting}
jblas::sym_mat mat_nn(100);
      \end{lstlisting}
    \end{onlyenv}
   \item<3-> Create symmetrix matrix from non symmetric matrix
    \begin{onlyenv}<3-3>
      \begin{lstlisting}
jblas::sym mat_nn(100);
jblas::sym_mat smat_nn =
    ublas::symmetric_adaptor<jblas::mat44,
        ublas::lower>( mat_nn );
jblas::sym_mat smat_nn =
    ublas::symmetric_adaptor<jblas::mat44,
        ublas::upper>( mat_nn );
      \end{lstlisting}
    \end{onlyenv}
   \item<4-> Access elements
    \begin{onlyenv}<4-4>
      \begin{lstlisting}
jblas::sym_mat22 mat_22;
mat_22(0,1) = 10.0;
// Warning:
mat_22(1,0) = 12.0;
JFR_DEBUG( mat_22(1,0) ); // will display 10.0 !
      \end{lstlisting}
    \end{onlyenv}
  \end{itemize}

\end{frame}

%%%%%%%%%%%%%%%%%%%%%%%%%%%%%%%%%%%%%%%%%%%%%%%%%%%%%%%%%%%%%%%%%%%%%%%%%%%%%%%%

\begin{frame}[fragile]
  \frametitle{Use lapack}
  \begin{itemize}
   \item<1-> To compute SVD and eigen values
   \item<2-> Warning: use column major with Lapack
      \begin{lstlisting}
jblas::mat A( 30, 3 );
jblas::mat_column_major m_A( A );
jblas::vec s(3);
jblas::mat_column_major U(30, 3);
jblas::mat_column_major VT(3, 3);

int ierr = lapack::gesdd(m_A,s,U,VT);
      \end{lstlisting}
  \end{itemize}
\end{frame}


%%%%%%%%%%%%%%%%%%%%%%%%%%%%%%%%%%%%%%%%%%%%%%%%%%%%%%%%%%%%%%%%%%%%%%%%%%%%%%%%

\begin{frame}[fragile]
  \frametitle{Linear least square}
  \begin{itemize}
    \item<1-> Find $ x $ that minimize $ \left\| A.x - b \right\|^2 $
    \item<2-> LinearLeastSquares
      \begin{onlyenv}<2-2>
        \begin{lstlisting}
LinearLeastSquares lls;
lls.setSize( 3 /* model size */,
             10 /* number of points */ );
jblas::vec valueOfA;
double valueOfB;
lls.setData( 0 /* index of point */,
             valueOfA,
             valueOfB );
...
lls.solve();
lls.x(); // return the value of x
lls.xCov(); // return the covariance
        \end{lstlisting}
      \end{onlyenv}
    \item<3-> VariableSizeLinearLeastSquares
      \begin{onlyenv}<3-3>
        \begin{lstlisting}
VariableSizeLinearLeastSquares vsll
            ( 3 /* model size */ );
jblas::vec valueOfA;
double valueOfB;
vsll.addMeasure( valueOfA, valueOfB );
...
vsll.solve();
vsll.x(); // return the value of x
        \end{lstlisting}
      \end{onlyenv}
  \end{itemize}

\end{frame}

%%%%%%%%%%%%%%%%%%%%%%%%%%%%%%%%%%%%%%%%%%%%%%%%%%%%%%%%%%%%%%%%%%%%%%%%%%%%%%%%
%%%%%%%%%%%%%%%%%%%%%%%%%%%%%%%%%%%%%%%%%%%%%%%%%%%%%%%%%%%%%%%%%%%%%%%%%%%%%%%%

\begin{frame}
  \frametitle{geom: Features}
  \begin{itemize}
   \item<1-> T3D: 3D Transformation
   \item<2-> Geometric classes such as Point, Lines, Boxes...
  \end{itemize}
\end{frame}

\begin{frame}[fragile]
  \frametitle{T3D: 3D Transformation}
  \begin{itemize}
    \item<1-> Support for Euler and Quaternion
      \begin{onlyenv}<1-1>
        \begin{lstlisting}
jblas::vec transfoX(6);
transfoX(0) = x;
transfoX(1) = y;
transfoX(2) = z;
transfoX(3) = yaw;
transfoX(4) = pitch;
transfoX(5) = roll;
geom::T3DEuler transfo( transfoX );
        \end{lstlisting}
      \end{onlyenv}
    \item<2-> Composition
      \begin{onlyenv}<2-2>
        \begin{lstlisting}
geom::T3DEuler robotToWorld = something;
geom::T3DEuler sensorToRobot = something;
geom::T3DEuler sensorToWorld;
geom::T3D::compose( sensorToRobot, 
                    robotToWorld,
                    sensorToWorld);
        \end{lstlisting}
      \end{onlyenv}
    \item<3-> Invert
      \begin{onlyenv}<3-3>
        \begin{lstlisting}
geom::T3DEuler robotToWorld = something;
geom::T3DEuler worldToRobot;
geom::T3D::invert( robotToWorld, worldToRobot );
        \end{lstlisting}
      \end{onlyenv}
    \item<4-> Transform a vector
      \begin{onlyenv}<4-4>
        \begin{lstlisting}
geom::T3DEuler sensorToWorld = something;
jblas::vec X_sensor;
jblas::vec X_world = ublas::prod( sensorToWorld.getM(), X_sensor );
        \end{lstlisting}
      \end{onlyenv}
    
  \end{itemize}
\end{frame}

%%%%%%%%%%%%%%%%%%%%%%%%%%%%%%%%%%%%%%%%%%%%%%%%%%%%%%%%%%%%%%%%%%%%%%%%%%%%%%%%

\begin{frame}[fragile]
  \frametitle{Geometric classes}
  \begin{itemize}
    \item<1-> Points
      \begin{onlyenv}<1-1>
        \begin{lstlisting}
jblas::vec v = coordinates of the point;
geom::Point<3> p1( 
          new Point<3>::EuclideanDriver( v ) );
        \end{lstlisting}
      \end{onlyenv}
    \item<2-> Lines
      \begin{onlyenv}<2-2>
        \begin{lstlisting}
jblas::vec v = coordinates of the origin;
jblas::vec v = coordinates of the vector director;
geom::Line<3> l1( 
            new Line<3>::EuclideanDriver( v ) );
geom::Point<3> p1;
geom::Point<3> p2;
geom::Line<3> l2(
    new Line<3>::TwoPointsPointerDriver( &p1, &p2 ) );
geom::Line<3> l2(
              new Line<3>::TwoPointsDriver( p1, p2 ) );
        \end{lstlisting}
      \end{onlyenv}
     \item<3-> Segments, polylines, planes, facets...
     \item<4-> Operations
      \begin{onlyenv}<4-4>
        \begin{lstlisting}
geom::distance( p1, p2 );
geom::distance( p1, l2 );
geom::angle( l1, l2 );
        \end{lstlisting}
      \end{onlyenv}
     \item<5-> Bounding box
\begin{onlyenv}<5-5>
  \begin{lstlisting}
geom::Segment segment( {1,1,1}, {-1,-1,-1} );
segment.boundingBox(); // Return {1,1,1}, {-1,-1,-1}
  \end{lstlisting}
\end{onlyenv}
  \end{itemize}
\end{frame}

\begin{frame}[fragile]
  \frametitle{Geometric classes : VoxelSpace (1/2)}

  \begin{lstlisting}
class Object
{
  public:
    Object(const geom::Atom<3>& atom_)
      : m_atom(atom_)
    {
    }
    const geom::Atom<3>& atom() const 
    { return m_atom; }
  private:
    const geom::Atom<3>& m_atom;
};
  \end{lstlisting}
\end{frame}

\begin{frame}[fragile]
  \frametitle{Geometric classes : VoxelSpace (2/2)}

  \begin{lstlisting}
geom::VoxelSpace<dimension, Object,
    geom::AtomBoundingBoxGetter<dimension, Object> >
    voxelSpace;
geom::Point<3> pt;
Object* obj1 = new Object(pt);
voxelSpace.insertObject( obj1 );
geom::Line<3> li;
Object* obj2 = new Object(li);
voxelSpace.insertObject( obj2 );
geom::BoundingBox<3> bb( onePoint, oneAnotherPoint );
std::list<Object*> objects =
                      voxelSpace.objectsIn( bb );
  \end{lstlisting}

\end{frame}

%%%%%%%%%%%%%%%%%%%%%%%%%%%%%%%%%%%%%%%%%%%%%%%%%%%%%%%%%%%%%%%%%%%%%%%%%%%%%%%%
%%%%%%%%%%%%%%%%%%%%%%%%%%%%%%%%%%%%%%%%%%%%%%%%%%%%%%%%%%%%%%%%%%%%%%%%%%%%%%%%

\begin{frame}
  \frametitle{image: Features}
  \begin{itemize}
   \item<1-> load/read images
   \item<2-> access to the whole OpenCV API\\
an \texttt{image::Image} object can be used as a \texttt{cv::Mat}
  \end{itemize}
\end{frame}

%%%%%%%%%%%%%%%%%%%%%%%%%%%%%%%%%%%%%%%%%%%%%%%%%%%%%%%%%%%%%%%%%%%%%%%%%%%%%%%%

\begin{frame}[fragile]
  \frametitle{Features}
  \begin{itemize}
   \item<1-> Create an image
\begin{lstlisting}
image::Image dx(width, height, CV_8U, JfrImage_CS_GRAY);
\end{lstlisting}
    \item<2-> Use a functin from OpenCV
\begin{lstlisting}
image::Image myImage;
myImage.loadImage("MyFile.png");
cvSobel(myImage, dx, 1, 0, 3);
\end{lstlisting}
  \end{itemize}

\end{frame}

%%%%%%%%%%%%%%%%%%%%%%%%%%%%%%%%%%%%%%%%%%%%%%%%%%%%%%%%%%%%%%%%%%%%%%%%%%%%%%%%

\begin{frame}
  \frametitle{camera: Features}
  Camera models:
  \begin{itemize}
   \item Pinhole, Barreto (omni), Stereo bench
   \item project, jacobians...
  \end{itemize}

\end{frame}

%%%%%%%%%%%%%%%%%%%%%%%%%%%%%%%%%%%%%%%%%%%%%%%%%%%%%%%%%%%%%%%%%%%%%%%%%%%%%%%%
%%%%%%%%%%%%%%%%%%%%%%%%%%%%%%%%%%%%%%%%%%%%%%%%%%%%%%%%%%%%%%%%%%%%%%%%%%%%%%%%

\begin{frame}
  \frametitle{datareader: Features}
  Ease the use of data respecting the organization of the pelican/tic server
  \begin{itemize}
    \item<1-> Read calibration infos
    \item<2-> Read images and preprocess them
    \item<3-> Read position infos
  \end{itemize}
\end{frame}

\begin{frame}[fragile]
  \frametitle{Default configuration}
    \begin{itemize}
      \item<1-> set the base directory
        \begin{lstlisting}
Jafar::Datareader::DataReader
    .setDefaultBasePath("~/laas/data")
        \end{lstlisting}
      \item<2-> set the series name
        \begin{lstlisting}
Jafar::Datareader::DataReader
    .setDefaultSeriesName("2010-11-28_grande-salle")
        \end{lstlisting}
      \item<3-> set the serie number
        \begin{lstlisting}
Jafar::Datareader::DataReader
    .setDefaultSerieNumber(11)
        \end{lstlisting}
    \end{itemize}
\end{frame}

\begin{frame}[fragile]
  \frametitle{Read data}
  \begin{lstlisting}
dr = Datareader::DataReader.new
sr = dr.getStereoReader( 0 )
img = sr.left.loadImage( 0 )
  \end{lstlisting}
\end{frame}

%%%%%%%%%%%%%%%%%%%%%%%%%%%%%%%%%%%%%%%%%%%%%%%%%%%%%%%%%%%%%%%%%%%%%%%%%%%%%%%%
%%%%%%%%%%%%%%%%%%%%%%%%%%%%%%%%%%%%%%%%%%%%%%%%%%%%%%%%%%%%%%%%%%%%%%%%%%%%%%%%

\begin{frame}
  \frametitle{qdisplay: Features}
  \begin{itemize}
   \item<1-> Uses QT
   \item<2-> Displays images
   \item<3-> Displays vector graphics overlay 
  \end{itemize}
\end{frame}

\begin{frame}[fragile]
  \frametitle{Use}
  \begin{lstlisting}
dr = Datareader::DataReader.new
sr = dr.getStereoReader( 0 )
imgL = sr.left.loadImage( 0 )
imgR = sr.right.loadImage( 0 )

viewer = Jafar::Qdisplay::Viewer.new
imageviewL = Jafar::Qdisplay::ImageView.new(imgL)
viewer.setImageView(imageviewL)
imageviewR = Jafar::Qdisplay::ImageView.new(imgR)
viewer.setImageView(imageviewR, 1, 0)

shape = Qdisplay::Shape.new(Qdisplay::Shape::ShapeRectangle,
  10, 10, 5, 5)
shape.setColor(0,255,0)
shape.setLabel("Hello World!")
imageviewL.addShape(shape)
  \end{lstlisting}
\end{frame}


\begin{frame}[fragile]
  \frametitle{gdhe: Features}
  \begin{itemize}
   \item<1-> Easy interface to be a client of GDHE
   \item<2-> GDHE objects are wrapped to C++ objects
  \end{itemize}

  Example:
  \begin{lstlisting}
gdhe::Client client;
client.connect("localhost");

gdhe::Ellipsoid *ell = new gdhe::Ellipsoid(x,xCov,3.);
ell->setLabel("12");
ell->setColor(255,0,0);
ell->setLabelColor(255,0,0);

client.addObject(ell);
  \end{lstlisting}


\end{frame}


%%%%%%%%%%%%%%%%%%%%%%%%%%%%%%%%%%%%%%%%%%%%%%%%%%%%%%%%%%%%%%%%%%%%%%%%%%%%%%%%
\subsection{Algorithms modules}

\begin{frame}
  \frametitle{Algorithms modules}
  Lower or higher level:
  \begin{itemize}
   \item Image Processing: correl, fdetect (harris, sift, surf, star), gfm, klt, dseg, jstereopixel...
   \item Optimization/Estimation: filter, jbn, localizer, bundle, ddf, oracle ...
   \item Localization/Modeling: slam, vme, dtm, ...
  \end{itemize}
  And yours !
\end{frame}


%%%%%%%%%%%%%%%%%%%%%%%%%%%%%%%%%%%%%%%%%%%%%%%%%%%%%%%%%%%%%%%%%%%%%%%%%%%%%%%%
%%%%%%%%%%%%%%%%%%%%%%%%%%%%%%%%%%%%%%%%%%%%%%%%%%%%%%%%%%%%%%%%%%%%%%%%%%%%%%%%
%%%%%%%%%%%%%%%%%%%%%%%%%%%%%%%%%%%%%%%%%%%%%%%%%%%%%%%%%%%%%%%%%%%%%%%%%%%%%%%%
\section{Conclusion}

\begin{frame}
  \frametitle{Conclusion}
  Jafar helps you:
  \begin{itemize}
   \item<1-> Provides framework and tools,
   \item<2-> Provides algorithms,
   \item<3-> Provides visibility to your software.
  \end{itemize}
  But you also have to help Jafar:
  \begin{itemize}
		\item<4-> before implementing, check if it already exists,
		\item<5-> put the stuff you create in the right module (functional separation),
		\item<6-> adopt standard coding: wiki://Jafar/Development/Rules\\
or more generally: http://www.possibility.com/Cpp/CppCodingStandard.html
		\item<7-> document your work
		\item<8-> write unit tests
  \end{itemize}
\end{frame}


\end{document}
